\documentclass[a4paper, 11pt]{article}
\usepackage[margin=1in]{geometry} % changes the margin
\usepackage{hyperref}
\renewcommand{\thefootnote}{\fnsymbol{footnote}}

%%%%%%%%%%%%%%%%%%%%%%%%%%%%%%%%%%%%%%%%%%%%%%%%%%%%%%%%%
%%%%%%%%%%%%%%%%%%%%%%%%%%%%%%%%%%%%%%%%%%%%%%%%%%%%%%%%%
%%%%%%%%%%%%%%%%%%%%%%%%%%%%%%%%%%%%%%%%%%%%%%%%%%%%%%%%%
\begin{document}
\pagenumbering{gobble}
%Header-Make sure you update this information!!!!
\begin{noindent}
\large\textbf{Reading Report - Week 1} \hfill \textbf{Varad Meru} \\
\normalsize CS 221 - Information Retrieval \hfill Student \# 26648958 \\
Prof. Cristina Lopes \hfill Due Date: 01/10/2015
\end{noindent}
\noindent\makebox[\linewidth]{\rule{\textwidth}{0.4pt}}
%%%%%%%%%%%%%%%%%%%%%%%%%%%%%%%%%%%%%%%%%%%%%%%%%%%%%%%%%
%%%%%%%%%%%%%%%%%%%%%%%%%%%%%%%%%%%%%%%%%%%%%%%%%%%%%%%%%
%%%%%%%%%%%%%%%%%%%%%%%%%%%%%%%%%%%%%%%%%%%%%%%%%%%%%%%%%
\begin{center}
\textbf{\Large{{As We May Think}\footnote{Published in \textit{The Atlantic Monthly}, July 01, 1945}}: A Summary}\\
By {Vannevar Bush}\footnote{\href{http://en.wikipedia.org/wiki/Vannevar_Bush}{Wikipedia - Vannevar Bush}} (1945)
\end{center}
\vspace{-25pt}
%%%%%%%%%%%%%%%%%%%%%%%%%%%%%%%%%%%%%%%%%%%%%%%%%%%%%%%%%
%%%%%%%%%%%%%%%%%%%%%%%%%%%%%%%%%%%%%%%%%%%%%%%%%%%%%%%%%
%%%%%%%%%%%%%%%%%%%%%%%%%%%%%%%%%%%%%%%%%%%%%%%%%%%%%%%%%
\section*{Summary}
\vspace{-5pt}
The author starts of with the scenario of the science community after the World War II and its profound impact on scientists from various fields, especially physicists. He then addresses the problem of scientists finding relevant information for their research in the ever-increasing work done by others in his/her field. Finding information that would support and extend the research is critical for the end goal of a successful work. He presents the then state-of-the-art methods of transmitting and reviewing the results and what they lack to help reduce the problem as mentioned earlier and also points out the technologies which were in the horizon, which would help in the task of recording, archiving and reproducing the details of research.

The discussion is then led to the specific work coming up such as \textit{Photocells}, \textit{advanced photography}, \textit{thermionic tubes}, \textit{cathode ray tubes}, \textit{relay combinations} and \textit{machine with interchangeable parts}, and others. He proposes the use of these technologies to store scientific records and help the scientists do it in a faster and better way than the then, current standard. He starts of explaining how \textit{dry photography} can be used to instantly generate photographs with some ideas on the use of ammonia gas to destroy the unexposed dye. Then, only wet photography was prevalent and it was a cumbersome process which took a lot of time to produce results. He then shifts his focus to \textit{microphotography}, which, like the dry photography, was not available in that period. A microphotograph would be used to store vast amount of information in a compressed form and read using projections.

He then moves to a more generic field of print media and its extensions with a text-to-speech (Voder) and speech-to-text (Vocoder) and presents an idea of an automatic stenotype with the use of Vocoder. He also envisions a future where an explorer has a personal camera and a Vocoder-like system in his laboratory and his commentary along with the photographs are synced using timestamps. After explaining his thoughts on next generation arithmetic machines, he goes on to explain its possible impact in a real-world scenario of counting, especially in accounting, retail systems, along with the capability of storage using cards or films. He then moves to presenting his ideas on logic representation for machines which would enable machine to deduce based on the logic symbols. 

Author hints on the problem of selection from the vast store of ideas from records and its manipulation. He expresses his ideas on the approach of retrieving the thoughts in a was automatic telephone exchange might be doing it. He then addresses the issue of with a different approach of association and introduces a device which would store all the books, records, and communications, and coined its name as \textit{memex}, which would supplement a researcher's memory and use indexing to retrieve stored documents faster. 

%%%%%%%%%%%%%%%%%%%%%%%%%%%%%%%%%%%%%%%%%%%%%%%%%%%%%%%%%
%%%%%%%%%%%%%%%%%%%%%%%%%%%%%%%%%%%%%%%%%%%%%%%%%%%%%%%%%
%%%%%%%%%%%%%%%%%%%%%%%%%%%%%%%%%%%%%%%%%%%%%%%%%%%%%%%%%
\vspace{-10pt}
\section*{Commentary}
\vspace{-5pt}
This article helped me get to know first-hand how futurists and visionaries of the previous generation have helped shape systems which we use now or will build in the future. I could see striking similarities in the systems which are currently available and the ones described by the author, such as Programming languages (Symbolic languages), Polaroid cameras (dry photographs), \textit{memex} (hand-held devices), speech recognition and many more.
\end{document}